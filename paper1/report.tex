

\documentclass[sigconf]{acmart}

\usepackage{hyperref}

\usepackage{endfloat}
\renewcommand{\efloatseparator}{\mbox{}} % no new page between figures

\usepackage{booktabs} % For formal tables

\settopmatter{printacmref=false} % Removes citation information below abstract
\renewcommand\footnotetextcopyrightpermission[1]{} % removes footnote with conference information in first column
\pagestyle{plain} % removes running headers

\begin{document}
\title{Amazon Web Services (AWS) in Support of Big Data and Analytics}


\author{Peter Russell}
\orcid{1234-5678-9012}
\affiliation{%
  \institution{University of Indiana - Bloomington}
}
\email{petrusse@iu.edu}

\begin{abstract}
This paper will explore the logistics of Amazon Web Services and how companies are currently utilizing the service to process their big data needs. 
\end{abstract}

\keywords{Big Data, Cloud Computing, AWS, Big Data Analytics}

\maketitle

\section{Introduction}

Amazon Web Services (AWS), the cloud service arm of Amazon, is currently the most dominant company in the cloud computing marketplace. With a market share of $31\%$, AWS holds a larger share than the next three closest competitors (Google, Microsoft and IBM)\cite{aws_mkt}. As a $\$10$ billion a year line of business for Amazon, the revenue stream is incredibly diversified across multiple product offerings. One of these categories, which can broadly be described as `business analytics,' have helped companies gain new insights into their customer experiences and competitive landscape. 


\bibliographystyle{ACM-Reference-Format}
\bibliography{report} 


\end{document}
